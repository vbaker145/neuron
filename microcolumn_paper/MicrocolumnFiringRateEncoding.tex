\documentclass[a4paper,11pt]{article}
\usepackage[utf8]{inputenc}
\usepackage{amsmath}
\usepackage{amsfonts}
\usepackage{amssymb}
\usepackage{graphicx}
\usepackage{tabularx}
\usepackage[font=scriptsize]{caption}
\usepackage[font=scriptsize]{subcaption}
\usepackage{wrapfig}
\usepackage[backend=biber]{biblatex}

\addbibresource{nn.bib}
\renewcommand\thesubsection{\alph{subsection}}


%opening
\title{Synchronization of traveling waves by low-threshold-spiking inihibitory neurons}
\author{Vince Baker, advisor: Dr. Luis Cruz Cruz\\ Drexel University Department of Physics}

\begin{document}

\maketitle

\begin{abstract}
We propose a neural circuit topology to encode population firing rate information.
The topology is based on the observed minicolumn structure found in the mammalian cortex.
The proposed neural circuit provides a strong, sparse encoding of peaks in the population firing rate through synchronized traveling waves in the minicolumn structures.
\end{abstract}

\section{Introduction} 

Neurons in the brain that fire sequentially can spread their firings to neighboring neurons creating what has been called a travelling wave of neuron activation. 
These travelling waves have been observed in the cortex of mammalian brains as well as in vitro, and subsequently have been reproduced in silico \cite{keane2015}

\section{Methods}

\section{Results}


\section{Discussion}


\clearpage
\printbibliography

\end{document}
