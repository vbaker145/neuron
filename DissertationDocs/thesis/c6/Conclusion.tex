%%
%% This is file `example-1.tex',
%% generated with the docstrip utility.
%%
%% The original source files were:
%%
%% drexel-thesis.dtx  (with options: `example-part')
%% 
%% This is a generated file.
%% 
%% Copyright (C) 2010 W. Trevor King
%% 
%% This file may be distributed and/or modified under the conditions of
%% the LaTeX Project Public License, either version 1.3 of this license
%% or (at your option) any later version.  The latest version of this
%% license is in:
%% 
%%    http://www.latex-project.org/lppl.txt
%% 
%% and version 1.3 or later is part of all distributions of LaTeX version
%% 2003/06/01 or later.
%% 

\chapter{Conclusion}

\section{Quasi 1-D results}
Here we have demonstrated that a biologically influenced model of a small quasi one-dimensional columnar ensemble  can support traveling waves. 
Our model considers a wide variety of neuronal types using local connectivity and distance-dependent time delays.
Because of the substantial randomness in neuronal types and connections between neurons, we did not observe traveling waves in the purely one-dimensional ($X=1,Y=1$) version of our model, 
in contrast to previous work that considered traveling waves in homogeneous neuronal populations in strictly one-dimensional systems. 

In our model various parameters determine traveling wave formation and propagation including the connectivity, connection strength and excitatory/inhibitory balance.
Columns with local connectivity support traveling waves, consistent with other simulations and experimental studies of cortical traveling waves \citep{Kopell1986}\citep{ermentrout2001}\citep{Golomb1999} .
The speed of these waves are influenced by connectivity, with a logarithmic dependence on connection strength that is consistent with previous work \citep{Golomb1996}\citep{Golomb1999}.

We have also shown that  a fully-connected SCE exhibits traveling waves if the action potential propagation time depends on the inter-neuron distance. 
From the dynamics point of view, we have shown that the action potential propagation delays affect propagation speed (Figure \ref{fig:delay_speed}) but is not a determinant on whether a wave can exist 
(Figure \ref{fig:wave_parameters}d).
We have also shown that  the connectivity and connection strengths influence the wave propagation speed (Figures \ref{fig:delay_k} and \ref{fig:delay_topology}) 
through the neuron dynamics (Figure \ref{fig:delay_neurondynamics}) .
From the point of view of neuronal types  we have determined that preferential sites for wave initiation are due to the presence of inhibitory neurons with low spiking thresholds.

Previous research into traveling waves in neuronal networks has identified parameter regions of neural-field models for which traveling waves exist \citep{Wilson1973}\citep{Ermentrout1979}. 
In these models critical values of model parameters are usually revealed by a linear stability analysis.
Our exploration of model parameters (Figure \ref{fig:wave_parameters}) instead shows a gradual transition region in which waves emerge and dominate the firing activity.
In \citep{Senk2020} the authors derive critical parameters using linear stability analysis of a neural field and find the critical parameter values, then map those parameters
to a network of leaky integrate-and-fire neurons.
Although the stability analysis shows sharply defined regions (their Figure 3 and 4), their spiking neuron simulation results show a gradual transition between regions (their Figure 7) similar to our work.

We have investigated the effect of incrementally increasing the dimensionality of our SCE on the speed and existence of traveling waves by considering a larger base (width described by X and Y) 
but still preserving its quasi one-dimensional character (ratio of Z to X or Y from 50 to greater than 7).
Our results show that the speed of the traveling waves initiated at the bottom of the SCE increased as a function of width (Figure \ref{fig:delay_topology}).
This can be understood by noting that we use open boundary conditions that when considering a larger width will inevitably increase the average number of connections per neuron in the SCE.
This means that neurons on average will receive a higher input current that, by its similarity to the mechanisms behind Figure \ref{fig:delay_k}, will cause a higher speed. 
We further show that the increased number of connections per neuron is the sole mechanism behind this effect by obtaining a constant speed when keeping constant the average number of connections
(Figure \ref{fig:delay_topology_avgconn}).
By using the same system with a constant number of connections per neuron, but now using a background stimulation to all neurons, we show that the value of the connection strength beyond which traveling waves (or synchronous firing) exist is very similar to that shown in Figure \ref{fig:wave_parameters}a, independent of width of SCE (SI Figure 5).
Moreover, the graphs of raster plots as a function of width seem to indicate that there is a value for the width at which excitations change from traveling waves to synchronous firing (SI Figure 5).
Future work will consider whether this change represents a phase transition between traveling waves and synchronous firing, a transition that is not uncommon to find in other systems but controlled by other intrinsic coupling parameters \citep{ermentrout2001}\citep{Senk2020}.

Other research \citep{keane2015} has shown that complex spatial patterns emerge from two-dimensional networks of leaky-integrate-and-fire neurons when the excitation and inhibition are balanced.
In that work, when excitation dominates then only plane waves were observed instead of complex patterns.
Similarly, in one dimension we have found that traveling waves dominate the firing activity when the $P_{exc}$ grows close to $1$ (Figure \ref{fig:wave_parameters}).
Both studies show that traveling waves can become the dominant spatial pattern in excitatory networks.

We have found that low-threshold-spiking inhibitory neurons can determine the origin point of traveling waves when using a uniform background stimulus (Figure \ref{fig:lts_inhibit}).
Parameter variations in the Izhikevich model \citep{izhikevich2003} create low-threshold spiking found in some inhibitory neurons \citep{gibson2009}\citep{hayut2011}, leading to post-inhibitory rebound spiking in the connected excitatory neurons \citep{ascoli2010}.
Our result is fundamentally different than previous studies of oscillatory behavior in populations of coupled excitatory and inhibitory neurons \citep{Golomb1996}\citep{Golomb1999} as that work considered homogeneous populations with isotropic connectivity.
\citet{Sessolo2015} found that stimulating inhibitory neurons in vitro could eithertrigger or inhibit epiletiform activity depending on the location of the stimulus.

Our result implies that, when no stimulus is applied, a biological neuronal network might have an equilibrium dynamics governed by traveling waves initiated by low-threshold-spiking inhibitory neurons.
An applied stimulus could then disturb the network from this dynamic equilibrium state.
Our results show  that any network with sufficiently strong connectivity and delayed action potential propagation can exhibit traveling wave structure (Figure \ref{fig:delay_waves}).
Since biological action potentials always travel with some delay, this indicates that traveling waves could be observed in the brain regardless of the inter-neuron connectivity.
The numerical results presented here could be further refined to predict biological traveling wave speed for different neuron and network parameters.
This could help distinguish between the different classes of traveling waves presented in \citet{ermentrout2001}.

Traveling waves in one-dimensional networks have not been observed in the brain. 
Part of the reason might be that such an observation requires precise experiments that probe structures such as minicolumns in the cortex. 
These are not easy, as other properties of minicolumns are still elusive. 
However, our work shows that in biological quasi one-dimensional systems  traveling waves are not only possible, but that because of the wide range of values in their allowed parameter space, their existence should be expected.

\section{Quasi-2D results}
We have demonstrated traveling waves in quasi two-dimensional sheets with substantial spatial variation in the neurons and synapses.
Previous two-dimensional simulations using neural field models or leaky integrate-and-fire neurons addressed systems of identical neurons and isotropic connections.
In our model a purely two-dimensional system did not exhibit traveling waves, although it did exhibit spatial and temporal correlations in the neural activity.

We have shown that our model can exhibit the circular spreading waves, plane waves and spiral waves that are observed in the cortex.
The spiral waves are most prevalent at lower connection strengths.
We have shown that a single spreading wave can fracture into a spiral wave due to the variation in the underlying neural system.

Our quasi-2D sheets exhibit repeating spatio-temporal wave patterns.
These patterns are fairly stable over time and we consider them as attractor states of the system.
A given sheet can exhibit multiple attractor states, as applying different random background stimulus to the same sheet results in different wave patterns.

Previous work \citep{keane2015} proposed traveling waves as an explanation for the seeming contradictions between the irregular spiking behavior of individual neurons 
and the correlation between the firing of nearby neurons.
We reproduce these results, showing substantial variance in the inter-spike intervals as well as higher correlation between the firing rates of nearby neurons.
We confirm that the neurons mostly fire when a traveling wave passes, but any individual neuron may not fire at all or fire multiple times when a particular wave passes.
We also find that for the special case of sparse connectivity our systems can exhibit extremely regular inter-spike intervals if each neuron fires only once when a wave passes.

We have also demonstrated traveling waves in a forest of minicolumns that represents the laminar and minicolumnar organization of the cortex.
The forest organization creates anisotropy between the vertical connectivity (down the minicolumn) and the lateral connectivity between minicolumns.
Transverse traveling waves are supported in the forest even when the lateral connectivity is lower than would normally permit wave propagation.
We observe both spreading and spiral waves structures depending on the connection strength.
The transverse wave speed is slower than the vertical wave speed within the minicolumns.


\section{Information transport}
We have proposed a mechanism for the encoding and transport of the information contained in the activity of a population of neurons via traveling waves of neuronal activation.
We demonstrated that our minicolumn structure can encode the population firing rate into the minicolumn wave rate.
The minicolumn wave rate depends asymptotically on the firing rate of the input population in a manner that resembles the activation functions of both single neurons and neural populations. 
Although the single minicolumn encoding was noisy, we demonstrated that a forest of loosely coupled  minicolumns could more reliably encode and transmit the population rate information.

\section{Future work}
The present work has established fundamental methods and results for exploring traveling waves in simulations of neuronal systems.
There are a number of areas for further research based on the results shown so far.

The present models do not include time-varying connection strengths or any model of synaptic plasticity.
A common form of spike-timing-dependent plasticity enhances a synaptic connection between two neurons when the presynaptic neuron fires shortly before the postsynaptic neuron fires.
Since traveling waves naturally impose a spatial sequence to the neuron firing activity it is reasonable to expect that synaptic plasticity could enhance traveling waves.
Traveling waves could even be an emergent behavior due to the long-term potentiation of synaptic pathways in a given direction.

We have observed that a given quasi two-dimensional system can settle into different attractor states.
It is possible that an external stimulus could cause a system in one attractor state to switch to a different attractor state.
Such stimuli-induced switching between stable states would be an interesting phenomenon with a number of possible functional roles.

The present work does not include any representation of the damage that may occur to neurons and synapses due to trauma, disease or aging.
Previous work \citep{cruz2000} has shown a loss of minicolumnar structure in the cortex of sufferers of Alzheimer's disease.
Our methods could be extended to examine how such changes to cortical organization could disrupt the formation and structure of traveling waves

We have shown that a small forest of minicolumns can both encode and transport information.
Such a structure might also perform a filtering function by e.g. selectively transmitting certain stimulus frequencies.
The coupling between adjacent minicolumns may also provide a mechanism for spatial filtering if different stimuli are applied to different minicolumns. 
Such an experiment is suggested by the observed columnar structure of the visual cortex in which neurons in the same column are tuned to the same stimulus.

Our minicolumn structure was inspired by the neuronal structure described in the seminal paper on Liquid State Machines \citep{maass2002}.
The structures presented here could be used as the 'liquid' or 'reservoir' in that paradigm of machine learning.
The waves observed in the quasi-two dimensional systems are reminiscent of the ripples in a pool of water from a falling stone, 
the analogy that inspired the term Liquid State Machine.

Interaction between temporal oscillations at different frequencies, termed cross-frequency coupling \citep{Hyafil2015}, has been observed in mammalian brain.
One widely studied example is the relationship between theta oscillations ($4-8~Hz$) and gamma oscillations ($>30~Hz$), 
which have been proposed as a means to represent multiple items in an ordered way \citep{Lisman2013}. 
The present work has produced traveling waves with repetition rates in the theta range.
This work could be extended with models that include higher-frequency oscillations to study the mechanisms and computational roles of cross-frequency coupling.

\endinput
%%
%% End of file `example-1.tex'.
