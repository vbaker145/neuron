%%
%% This is file `example-1.tex',
%% generated with the docstrip utility.
%%
%% The original source files were:
%%
%% drexel-thesis.dtx  (with options: `example-part')
%% 
%% This is a generated file.
%% 
%% Copyright (C) 2010 W. Trevor King
%% 
%% This file may be distributed and/or modified under the conditions of
%% the LaTeX Project Public License, either version 1.3 of this license
%% or (at your option) any later version.  The latest version of this
%% license is in:
%% 
%%    http://www.latex-project.org/lppl.txt
%% 
%% and version 1.3 or later is part of all distributions of LaTeX version
%% 2003/06/01 or later.
%% 

\chapter{Introduction}
Connected neurons that fire sequentially can stimulate firings from neighboring neurons, creating a traveling wave of neuron spikes. 
These traveling waves of neuronal activation have been observed in the cortex as reviewed in \citet{Muller2018}.
Trveling waves have been observed in mammalian brains \citep{Muller2018}\citep{reimer2010}  as well as in vitro \citep{wu2008}\citep{huang2004}\citep{Golomb1999}, and subsequently have been reproduced in silico \citep{keane2015}\citep{Senk2020}\citep{Golomb1996}\citep{ermentrout2001}. 
Various functional roles have been proposed for these traveling waves including spatiotemporal processing in the visual cortex \citep{wu2008}\citep{Muller2014}, place field coordination in the hippocampus \citep{lubernov2009}, and memory consolidation during sleep \citep{Dickey2021}.
Traveling waves have been shown to carry information in the  motor cortex \citep{Rubino2006} and visual cortex \citep{Besserve2015}.
These waves can provide synchronous activations of a population to serve functions such as gating perception of low-contrast images \citep{Davis2020}.
In addition to extensive observations in vivo, traveling waves have also been observed in neuronal network simulations at the mesoscopic level in simple one-dimensional systems \citep{Wilson1973}\citep{Golomb1999} and two-dimensional sheets \citep{keane2015}.
Traveling waves have also been studied in large-scale simulations of the entire brain \citep{Roberts2019}.

Neuronal traveling waves are not uniquely defined as they can exist at widely differing frequency and scale lengths.  
In the sleep and anesthetized states,slow and large-scale wave activity has been recorded \citep{Muller2018}. 
Traveling waves have been detected in local-field potentials measured with electrodes \citep{Rubino2006}\citep{sanes1993}\citep{Riehle2013}.
Local-area traveling waves in the awake cortex have been observed using recent advances in optical imaging techniques using voltage-sensitive dyes \citep{wu2008}\citep{Shoham1999}\citep{Xu2007}\citep{Ferezou2006}.  
Similarly, traveling waves also differ by the manner in which they are created. 
Both spontaneously occurring waves and stimulus-evoked \citep{reimer2010} waves have been observed in the visual and auditory cortex. 
Computational and mathematical modeling of traveling waves in a variety of models have shown many of the characteristics observed experimentally \citep{ermentrout2001}\citep{keane2015}\citep{gibson2009}.



\endinput
%%
%% End of file `example-1.tex'.
