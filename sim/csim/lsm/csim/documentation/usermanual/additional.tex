\subsection{How the network is simulated}

\csim employs a fixed time step simulation scheme for the
integration of the differential equations involved (represented by
objects). Currently all objects use the exponential Euler method with
the same integration time step (see \sect{sec:gf}).

During each time step a certain method (\texttt{advance}) is called
for each object part of the simulation. There is a fixed schedule
which depends a) on the class of the object and b) on the time of
creation. Here is the order of the classes.
%
\begin{enumerate}
\item Neuron
\item Synapse
\item Recorder
\end{enumerate}
%
This means that all objects of classes derived from a generic Neuron
are advanced first followed by objects of classes derived from a generic
Synapse and so on.

Note that for example a more detailed neuron model (like the
\secref{classCbNeuron}{CbNeuron}) has to advance all its child objects
like ion channels.

After a \texttt{csim('simulate',...)} command returns the network is
kept in its current state at this time $t$. A further
\texttt{csim('simulate',...)}  command continues the simulation at
this time $t$. Use \texttt{csim('reset');} to reset the simulation to
time $t=0$.

Note that recorded traces you get by the appropriate
\secref{cmd:get}{get command} always start at time $t=0$.


\Subsection{Global fields}{sec:gf}

There are a few fields which determine the global (i.e. not object
specific) behavior of \csim which can be modified via the
\secref{cmd:set}{set command}:
%
\begin{tabbing}
\quad\tt>> csim('\hyperlink{cmd:set}{set}',<fieldname>,<value>); \\
\end{tabbing}
%

\subsubsection*{Read/writeable fields}


\begin{description}

\item[dt] The integration time step used by all objects

\item[randSeed] The seed value for the random number generator

\item[verboseLevel] How much stuff should be written to the
  console window (no effect yet!).
  
\item[nThreads] Number of threads to use for simulation (no
  effect yet!). The multi-threaded version of \csim is currently under
  development.

\item[spikeOutput] Flag: 1 ... always output spikes as last cell element, 0 do not
  output spikes (obsolete).

\end{description}

\subsubsection*{The following fields are read-only:}

\begin{description}

\item[t] The current virtual (simulation) time
  
\item[step] The current time step; i.e.  \texttt{step=t/dt};

\end{description}

\subsection{Event driven simulation}

Since in a typical simulation the firing rate of the neurons is about
30\,Hz and the time constant of a synaptic current is typically about
3\,ms most of the time a synapse is ``idle''. This observation
encouraged us to implement some ideas of event driven simulators like
\href{http://www.cnl.salk.edu/~arno/spikenet/}{SpikeNet}\latex{\footnote{\url{http://www.cnl.salk.edu/~arno/spikenet/}}}:
We decided to cut off the very far tail of the exponential decaying
synaptic responses (after 5 time constants) and remove those synapses
from the list of active synapses (which are advanced every time step).
A synapse is re-activated if it receives a further spike. For large
networks with lots of synapses these idea results in an average
speed-up of 2 to 3.

