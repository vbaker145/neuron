
The common form of all \csim commands is 
\begin{verbatim}
  csim(command,...);
\end{verbatim}
where \texttt{command} is one of the following strings:

\begin{itemize}
\setlength{\itemsep}{-0.8ex plus 0.1ex minus 0.1ex}
\setlength{\parsep}{0.0ex plus 0.1ex minus 0.1ex}
\item \texttt{\secref{cmd:create}{'create'}} : Create an object of a certain class
\item \texttt{\secref{cmd:set}{'set'}} : Set fields of an object
\item \texttt{\secref{cmd:connect}{'connect'}} : Connect objects
\item \texttt{\secref{cmd:get}{'get'}} : Get field values of an object
% \item \texttt{\secref{cmd:access}{'access'}} : Set default network
\item \texttt{\secref{cmd:reset}{'reset'}} : Reset the simulation to $t=0.0$
\item \texttt{\secref{cmd:simulate}{'simulate'}} : Simulate the network
\item \texttt{\secref{cmd:export}{'export'}} : Export the network for saving it
\item \texttt{\secref{cmd:import}{'import'}} : Import a loaded network
\item \texttt{\secref{cmd:destroy}{'destroy'}} : Destroy/delete the current netork
\item \texttt{\secref{cmd:list}{'list'}} or \texttt{\secref{cmd:list}{'ls'}} List various items
\item \texttt{\secref{cmd:version}{'version'}} : Print out a version string
% \item \texttt{\secref{cmd:pvm}{'pvm'}} : Initialize distributed simulation
\end{itemize}

A detailed description of each of the commands follows. 

%\subsection{Multiple Networks}
%
%\csim is capable of holding more then one network at a time. There are
%two ways of specifing on which network a command should operate.
%
%\begin{itemize}
%\item For each command of the form \texttt{csim(command,...);} the
%form \texttt{csim(netIdx,command,...);} exists which allows you to
%explicitly specify the network to operate on.
%
%\item If \texttt{csim(command,...);} is used the \emph{default
%    network} is used. The default network is set using the command
%  \texttt{csim('access',netIdx);}.
%\end{itemize}
%
%\cmdref{access}
%
%\begin{itemize}
%  
%\item \texttt{idx=csim('access',netIdx);} makes the network specified
%  by \texttt{netIdx} the default network and returns that value.%
%
%\item \texttt{csim('access',netIdx);} makes the network specified
%  by \texttt{netIdx} the default network and prints that value.
%
%\item \texttt{idx=csim('access');} returns the index of the default
%  network.
%
%\item \texttt{csim('access');} prints the index of the default%
%
%\end{itemize}


\subsection{Setting up a network simulation}

\cmdref{create}

\begin{itemize}
  
%\item \texttt{idx=csim('create','Network');} creates a new network and
%  returns the \emph{index} or \emph{handle} for the created network.
%  The new network is the new default network; see command
%  \secref{cmd:access}{access}.

\item \texttt{idx=csim('create',class);} creates an object of the class
  specified by the string \texttt{class} and returns a \texttt{uint32}
  \emph{index} or \emph{handle} for the created object. This
  index/handle is the only means to access the created object.
  
\item \texttt{idx=csim('create',class,N);} creates \texttt{N} objects
  of the class specified by the string \texttt{class} and returns a
  \texttt{uint32} vector of indices/handles for the created objects.

\end{itemize}

The handles returned by \texttt{idx=csim('create',...);} are used to
identify the objects later in \texttt{'set'}, \texttt{'get'} and
\texttt{'connect'} calls.

See the \secref{sec:clref}{Class Reference} for a full description of
all available classes.

\cmdref{set}

\begin{itemize}
  
\item \texttt{csim('set',field,value);} sets the
  \secref{sec:gf}{network field} specified by the string
  \texttt{field} to \texttt{value}.
  
\item \texttt{csim('set',idx,field,value);} sets the field specified
  by the string \texttt{field} of the object with index/handle
  \texttt{idx} to \texttt{value}. \texttt{idx} can also be a vector
  of indices/handles of objects of the same class.
  
If \texttt{idx} is a vector and \texttt{value} is a scalar then the
fields of all objects are set to \texttt{value}.

If \texttt{idx} and \texttt{value} are vectors of the same size then
the field \texttt{field} of all objects \texttt{idx(i)} is set to
\texttt{value(i)}.

\item \texttt{csim('set',idx,f1,v1,f2,f2,...,fn,vn);} sets the fields
  specified by the strings \texttt{f1}, \texttt{f2}, ..., \texttt{fn}
  of the object with index/handle \texttt{idx} to the values
  \texttt{v1}, \texttt{v2}, ... \texttt{vn}. \texttt{idx} and
  \texttt{v1}, \texttt{v2}, ... \texttt{vn} can also be vectors.

\end{itemize}


\cmdref{connect}

\begin{itemize}
  
\item\texttt{csim('connect',dstIdx,srcIdx);} sets up a signal flow
  from the source \texttt{srcIdx} object (e.g. a synapse) to the
  destination object \texttt{dstIdx} (e.g. the postsynaptic neuron)
  where \texttt{dstIdx} and \texttt{srcIdx} are indices/handles to
  objects.
  
  \texttt{dstIdx} and \texttt{srcIdx} can also be vectors. In that
  case for each \texttt{i=1:length(srcIdx);} a signal flow
  $\mathtt{dstIdx(i)} \leftarrow \mathtt{srcIdx(i)}$ is set up.
  
\item\texttt{csim('connect', dstIdx, srIdx, viaIdx);} for each
  \texttt{i=1:length(srcIdx);} a signal flow of the form
  $\mathtt{dstIdx(i)} \leftarrow \mathtt{viaIdx(i)} \leftarrow
  \mathtt{srcIdx(i)}$ is set up where \texttt{dstIdx},
  \texttt{srcIdx}, and \texttt{viaIdx} are vectors of the same length
  of indices/handles to objects.
  
\item\texttt{csim('connect',recIdx,objIdx,fieldName);} connects the
  object with handle \texttt{objIdx} to the recorder with handle
  \texttt{recIdx}. The recorder \texttt{recIdx} will then record a
  trace of the values of the field \texttt{fieldName} of the objects
  specified by the vector of hadles \texttt{objIdx}.

\end{itemize}

\subsection{Running the network simulation}

\cmdref{reset}

\texttt{csim('reset');} resets the simulation time $t$ back to
$t=0.0$.

\Subsubsection{simulate}{cmd:singlesim}
\label{cmd:simulate}

\begin{itemize}
  
\item \texttt{csim('simulate',Tsim,InputSignals);} runs the network
  simulation for \texttt{Tsim} seconds starting at the time where the
  last simulate command stopped with \secref{sec:inputs}{input signals
    \texttt{InputSignals}}.
  
\item \texttt{csim('simulate',Tsim,I1,I2,...,In);} same as above but
  with input signals \texttt{I1} to \texttt{In}. There is no special
  meaning in the ordering of the input signals. It is equvivalent to
  concatenate \texttt{I1} to \texttt{In} into one struct array and
  pass this array as the single input signal.
  
\item \texttt{R=csim('simulate',Tsim,InputSignals);}\\ same as
  \texttt{csim('simulate',Tsim,InputSignals);} but in addition returns
  the cell array \texttt{R} which holds the \secref{sec:output}{output
    (traces) of all Recorder objects}.
  
\item \texttt{R=csim('simulate',Tsim,I1,I2,...,In);}\\ same as
  \texttt{csim('simulate',Tsim,I1,I2,...,In);} but in addition returns
  the cell array \texttt{R} which holds the \secref{sec:output}{output
    (traces) of all Recorder objects}.

\item \texttt{R=csim('simulate',Tsim,I1,I2,...,In);}\\ same as
  \texttt{csim('simulate',Tsim,I1,I2,...,In);} but in addition returns
  the cell array \texttt{R} which holds the \secref{sec:output}{output
    (traces) of all Recorder objects}.


\end{itemize}

%\Subsubsection{Multi-stimulus simulate}{cmd:singlemulti}
%
%\texttt{R=csim('simulate',Tsim,S);} runs $n$ network simulations for
%\texttt{Tsim} sec. The network is reset to $t=0$ before each
%simulation.
%
%The $n$ simulations are determined by the $n \times 1$ strucht array
%\texttt{S} where \texttt{S(i).channel} specifies the stimulus for the
%$i$-th simulation (see \sect{sec:multiin}).
%
%\texttt{R} contains the reorded values of all $n$ simulations (see
%\sect{sec:multiout}).
%  
%If the \secref{sec:distr}{distributed simulation} is on (see command
%\secref{cmd:pvm}{pvm}), the $n$ simulations will be carried out in
%parallel.
%
%\subsection{Initializing PVM}
%
%\cmdref{pvm}

\subsection{Saving, loading and deleting networks }

\cmdref{export}

The command
\begin{verbatim}
  net = csim('export');
\end{verbatim}
produces a struct array \texttt{net} which contains all information
about the current network simulation which is needed to set up the
simulation. Using \texttt{save file.mat net} you can save the whole
network simulation.

NOTE: The struct array \texttt{net} returned by
\texttt{net=csim('export');} is only intended to be saved to disk for
later use by \texttt{csim('import',net);} and does not have a human
readable form.

\cmdref{import}

The command
\begin{verbatim}
  csim('import',net);
\end{verbatim}
allows you to set up a network simulation from the struct array
\texttt{net} which has previously been generated by \texttt{net =
  csim('export');}. In combination with the matlab command
\texttt{load} this \csim command serves the purpose of loading a
simulation from disk.

If the current network (either explicitly specified or default
network) does not contain any objects the \texttt{net} will be
imported into that network. If there are already objects a new network
will be created and and \texttt{net} will be imported to that.

NOTE: The struct array \texttt{net} returned by
\texttt{net=csim('export');} \emph{does not} contain information about
the current state of the simulation at time $t$.  \emph{Hence it is
  only possible to start simulation based on such an arry at time
  $t=0.0$.}

\cmdref{destroy}

\texttt{csim('destroy');} all the networks.

\subsection{Displaying/Getting information}

\cmdref{get}

\begin{itemize}

  \item \texttt{csim('get');} diplays the values of the \secref{sec:gf}{global fields}

  \item \texttt{v=csim('get',field);} return the value of the
    \secref{sec:gf}{global field} specified by the string \texttt{field}

  \item \texttt{csim('get',idx);} displays the values of all fields of the
    object specified by the array index/handle \texttt{idx}

  \item \texttt{v=csim('get',idx,field);} returns a double array
   \texttt{v} where \texttt{v(:,i)} contains the values of the field
   \texttt{field} of the object with index/handle \texttt{idx(i)}.

 \item \texttt{o=csim('get',idx,'struct');} return a struct array
   describing the object specified by the index/handle
   \texttt{idx(1)}:

   \begin{itemize}
   \item \texttt{o.className} : String containing the name of class of object
   \item \texttt{o.spiking} : Double value which is 1 if the object is a
     spike emitting object (0 otherwise)
   \item \texttt{o.fields} : Cell array of strings containing the field
     names of the object
   \item and all fields with its values.
   \end{itemize}
   
 \item \texttt{[incomming,outgoing]=csim('get',idx,'connections');}
   returns the indices/handles of the incomming (i.e. input
   delivering) and outgoing (i.e. output receiving) objects of the
   object specified by the index/handle \texttt{idx}. This is
   currently only implemented for Synapses and Neurons.
   
 \item \texttt{R=csim('get',recorder\_idx,'traces');} returns the
   traces of the fields recorded by the
   \secref{classRecorder}{Recorder} object with handle
   \texttt{recorder\_idx} during the \emph{last simulation}. See the
   \secref{classRecorder}{Recorder documentation} for details about
   the format of \texttt{R}.

\end{itemize}

\cmdref{list}

\begin{itemize}
  
\item \texttt{csim('list');} and \texttt{csim('list','classes');}
  print a list of all available classes
  
\item \texttt{csim('list','classes','-fields');} prints a list of all
  available classes with information about the fields of each class.
  Read/writeable fields are marked by a '\texttt{=}' between the field
  name and its description, wheres a '\texttt{:}' denotes a read-only
  field.

\item \texttt{csim('list','objects');}  prints a list of all created
  objects

\item \texttt{csim('list','objects','-fields');} prints a list of all
  created objects and the values of its fields. Read/writeable fields
  are marked by a '\texttt{=}' between the field name and its value,
  wheres a '\texttt{:}' denotes a read-only field.

\item \texttt{csim('list','networks');}  prints a list of all networks

\item \texttt{csim('list','networks','-fields');} prints a list of all
  networks and the values of its fields. Read/writeable fields
  are marked by a '\texttt{=}' between the field name and its value,
  wheres a '\texttt{:}' denotes a read-only field.

\end{itemize}

Not that instead of \texttt{'list'} one can use \texttt{'ls'} as a
shortcut.

\cmdref{version}

\begin{itemize}
\item \texttt{csim('version')} prints a version string
\item \texttt{v=csim('version');} returns the version string
\end{itemize}

%%% Local Variables: 
%%% mode: latex
%%% TeX-master: t
%%% End: 
