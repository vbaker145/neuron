\Subsection{Input Signals}{sec:inputs}

% \Subsubsection{Single-stimulus input}{sec:singlein}

When runing a network simulation via \texttt{csim('simulate',...);}
one can specify \emph{input signals} like in the command
%
\begin{tabbing}
\quad\texttt{>> csim(\hyperlink{cmd:simulate}{'simulate'},Tsim,S);}
\end{tabbing}
%
where \texttt{S} is a struct array with the following fields:
%
\begin{itemize}
  
\item \texttt{S(i).idx} : array of handles of objects which should
  receives this signal

\item \texttt{S(i).spiking} : binary flag (0/1) which determines if
  \texttt{S(i).data} should be interpreted as spike times or as an
  analog signal
  
\item \texttt{S(i).dt} : time discretization; for analog signals
  (\texttt{S(i).spiking}=0) only
  
\item \texttt{S(i).data} : signal data: vector of the analog values
  (\texttt{S(i).spiking}=0) or spike times (\texttt{S(i).spiking}=1)

\end{itemize}
%
Note that \texttt{csim('simulate',...);} accepts an arbirary number of
such struct arrays. See \secref{cmd:simulate}{documentation of the
  simulate command}.


%\Subsubsection{Multi-stimulus input}{sec:multiin}
%
%When runing a network simulation via \texttt{csim('simulate',...);}
%one can specify a \emph{multi-stimulus} input \texttt{S} like in the command
%
%\begin{tabbing}
%\quad\texttt{>> R = csim(\hyperlink{cmd:simulate}{'simulate'},Tsim,S);}
%\end{tabbing}
%
%where \texttt{S} is a $n \times 1$ struct array of the form
%\texttt{S(i).channel} which describes the $i$-th stimulus.
%\texttt{S(i).channel} has the structure as described above.
%
%If such multi-stimulus input is specified \csim will simulate the
%current network with each of the $n$ stimuli where at the beginning of
%each stimuli the network is reset to $t=0$. The return argument
%\texttt{R} contains the recorded information of all $n$ simulations
%(see \sect{sec:multiout}).


\Subsection{Output}{sec:output}

If one simulates a network then one also wants to record the
quantities of interest. In our \secref{sec:start}{introductory
  example} we were interested in the membrane potential of the
leaky-integrate and fire neuron. \csim can record any field of any
object by means of \secref{classRecorder}{Recorder} objects. The
following code fragment shows how to set up a Recorder to record the
membrane potential (\texttt{Vm}) of a
\secref{classLifNeuron}{LifNeuron} object with handle \texttt{n}.
%
\begin{tabbing}
\quad\texttt{>> rec = csim(\hyperlink{cmd:create}{'create'},'\hyperlink{classRecorder}{Recorder}');} \\
\quad\texttt{>> csim(\hyperlink{cmd:connect}{'connect'},rec,n,'Vm');}
\end{tabbing}
%
Note that one Recorder object can record an arbitrary number of
fields from arbitrary objects.

\Subsubsection{Getting results of the last simulation}{sec:lastout}

After a network simulation via a command like
%
\begin{tabbing}
\quad\texttt{>> csim(\hyperlink{cmd:simulate}{'simulate'},Tsim,\hyperlink{sec:input}{InputSignal});}
\end{tabbing}
%
the recorded data of the \emph{the last simulation} Recorder
\texttt{rec} can be obtained by the command
\begin{tabbing}
\quad\texttt{>> R=csim(\hyperlink{cmd:get}{'get'},rec,'traces');}
\end{tabbing}
The exact structure of \texttt{R} for the recorder \texttt{rec}
depends on the value of the \secref{classRecorder}{field
\texttt{commonChannels} of the \texttt{Recorder}} object.


\paragraph{\texttt{commonChannels = 0}} In this case  (default) \texttt{R}
is a struct array with the only field \texttt{channel} which is in
turn a struct array with a similar structure as an
\secref{sec:inputs}{input signal}. That is

\begin{itemize}
  
\item \texttt{R.channel(j).idx} : handle of the object from which
  field the data was recorded
  
\item \texttt{R.channel(j).spiking} : binary flag (0/1) which
  determines if \texttt{data} should be interpreted as spike times or
  as an analog signal
  
\item \texttt{R.channel(j).dt} : time discretization; for analog
  signals   only
  
\item \texttt{R.channel(j).data} : signal data : vector of the
  analog values or spike times. Note that the data always starts at
  time $t=0$.
 
\item \texttt{R.channel(j).fieldName} : name of the recorded
    field

\end{itemize}

\paragraph{\texttt{commonChannels = 1}} In this case
\texttt{R} has two fields:
\begin{itemize}
\item \texttt{R.data} : A double array where
  \texttt{R.data(j,s)} is the $s$-th recorded value of the $j$-th
  field recorded.  Note that the data always
  starts at time $t=0$.  {\small Remark: The number of values which
    are recorded depends on the \secref{classRecorder}{field
      \texttt{dt} of the \texttt{Recorder} object}.}
  
\item \texttt{R.info} : A struct array where
  \texttt{R.info(j).idx} is the handle of the object from which
  the field \texttt{R.info(j).fieldName} is recorded.
\end{itemize}

\Subsubsection{Getting the results of a multi-stimulus simulation}{sec:multiout}

{\bf Reminder}: The command
\begin{tabbing}
\quad\texttt{>> R=csim(\hyperlink{cmd:get}{'get'},rec,'traces');}
\end{tabbing}
returns only the results of the last simulation. In the case of a
multi-stimulus simulation with stimulus array \texttt{S} of length $n$
this command returns only the result of the simulation with stimulus
\texttt{S(n)}.

To get the results of all $n$ simulations one has to use
\begin{tabbing}
\quad\texttt{>> R=csim(\hyperlink{cmd:simulate}{'simulate'},Tsim,S);}
\end{tabbing}

In that case \texttt{R} contains the results of all $n$ stimulations.
\texttt{R\{r\}(s).channel} contains the recorded data of the $r$-th
recorder during the $s$-th simulation ($1 \leq s \leq n$).
\texttt{R\{r\}(s).channel} is of the format as described in
\sect{sec:lastout}.




